\documentclass[a4paper]{article}

\usepackage[T1]{fontenc}
\usepackage[italian]{babel}
\usepackage[latin1]{inputenc}
\usepackage{graphicx}
\usepackage{float}
\usepackage[margin=2 cm]{geometry}
\usepackage{multirow}
\usepackage{multicol}
\usepackage{textcomp}
\usepackage{caption}
\author{Alberto Bordin, Giulio Cappelli}
\title{Duplicatore di frequenza}
\date{14-15 Dicembre 2017}
\newcommand{\minitab}[2][l]{\begin{tabular}#1 #2\end{tabular}}


\begin{document}
	\maketitle
	
	\begin{abstract}
		 
	\end{abstract}
	
\section{Teoria}

\section{Apparato sperimentale}

\section{Taratura}

1 tabella e 1 plot

\section{Segnale duplicato in funzione della potenza incidente}

\subsection{Presa dati}

1 tabella

\subsection{Analisi dati}

2 fit ax$^b$ e ax$^2$

\section{Angolo di phase-matching}

\subsection{Presa dati}

1 tabella

\subsection{Analisi dati}

1 fit a tutti a dati, 1 fit con un cut sulle code e 2 fit con gaussiana e parabola

\section{Polarizzazione}

\subsection{Presa dati}

\subsubsection{Polarizzazione in ingresso}

1 tabella

\subsubsection{Polarizzazione in uscita}

1 tabella

\subsection{Analisi dati}

\subsubsection{Polarizzazione in ingresso}

1 fit

\subsubsection{Polarizzazione in uscita}

1 fit

\newpage
\section*{Appendice}
	
\end{document}