\documentclass[a4paper]{article}

\usepackage[T1]{fontenc}
\usepackage[italian]{babel}
\usepackage[latin1]{inputenc}
\usepackage{graphicx}
\usepackage{float}
\usepackage[margin=2 cm]{geometry}
\usepackage{multirow}
\usepackage{multicol}
\usepackage{textcomp}
\usepackage{caption}
\author{A. Bordin, G. Cappelli}
\title{Visibile}
\date{4-5 Dicembre 2017}
\newcommand{\minitab}[2][l]{\begin{tabular}#1 #2\end{tabular}}


\begin{document}
	\maketitle
	
	\begin{abstract}
		 
	\end{abstract}
	
\section{Teoria}

\section{Apparato sperimentale}

\section{Presa dati}

\subsection{Taratura attenuatore}

Per prima cosa tariamo l'attenuatore in modo tale da avere la curva di conversione fra angolo e potenza. 
L'attenuatore � montato su di un goniometro cos� da permetterci di registrare attraverso il power meter l'uscita del laser di pompaggio variando l'angolo.

Dato che il laser � nell'infrarosso  ($\lambda$=975.9(2.9) nm) utilizziamo una cartina che rivela l'IR per allineare il laser nel power meter che abbiamo posto a $\sim$1 cm dall'attenuatore.



\subsection{Acquisizione segnale}

2 tabelle

\subsection{Acquisizione spettro}

\section{Analisi dati}

1 plot loglog con fit per alte e basse potenze imponendo l'esponente con 2 subplot residui

\subsection{Bassa potenza}

2 fit: ax$^b$ e ax$^2$

\subsubsection{Analisi del cut}

grafico di b in funzione del cut

\subsection{Alta potenza}

1 fit lineare

\subsection{Spettro di emissione}

plot spettro

\newpage
\section*{Appendice}
	
\end{document}