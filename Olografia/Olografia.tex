\documentclass[a4paper]{article}

\usepackage[T1]{fontenc}
\usepackage[italian]{babel}
\usepackage[latin1]{inputenc}
\usepackage{graphicx}
\usepackage{float}
\usepackage[margin=2 cm]{geometry}
\usepackage{multirow}
\usepackage{multicol}
\usepackage{textcomp}
\usepackage{caption}
\author{Alberto Bordin, Giulio Cappelli}
\title{Olografia}
\date{13-17 Novembre 2017}
\newcommand{\minitab}[2][l]{\begin{tabular}#1 #2\end{tabular}}


\begin{document}
	\maketitle
	
	\begin{abstract}
		 
	\end{abstract}

\begin{multicols}{2}	
\section{Teoria}
L'olografia � basata sulla registrazione di fase e ampiezza di un'onda elettromagnetica $a(x,y)= |a(x, y)|e^{i\phi(x,y)}$ su di un supporto che giace nel piano x-y, in modo da poter poi ricostruire l'immagine. Il supporto in questione � una lastra fotografica sensibile all'intensit� incidente, quindi � necessaria un'onda di riferimento $A(x,y)= |A(x, y)|e^{i\psi(x,y)}$. Infatti l'interferenza tra il fascio di riferimento e il fascio diffuso dall'oggetto provoca un pattern d'interferenza che contiene tutta l'informazione su ampiezza e fase in quanto l'intensit� incidente sulla lastra risulta essere:
\begin{equation}
I(x,y)=|A(x,y)+a(x,y)|^2= |A(x, y)|^2+|a(x, y)|^2+|a|^2+aA^*+Aa^*
\end{equation}
Assumendo che la trasmettivit� della lastra sia lineare con l'intensit� incidente, $t=\gamma I$, si ha:
\begin{equation}
t(x,y)=t_{ref}+\gamma(|a|^2+aA^*+Aa^*)
\end{equation}
dove $t_{ref} = \gamma|A|^2$ � la trasmettivit� di background dovuta all'onda di riferimento. 

\'E quindi possibile ricostruire l'immagine registrata utilizzando in trasmissione un'onda di ricostruzione $B(x,y)$, infatti il campo trasmesso �:
\begin{equation}
E_{trasm}=Bt=Bt_{ref}+\gamma(B|a|^2+BaA^*+BAa^*)=U_1+U_2+U_3+U_4
\end{equation}

Utilizzando come fascio di ricostruzione lo stesso fascio di luce usato per il riferimento e assumendo che $|A|^2$ sia costante su tutta la lastra, si hanno due termini lineari rispettivamente in $a$ e $a^*$ che contengono tutta l'informazione sull'oggetto in esame:
\begin{equation}
U_3=\gamma|A|^2a \qquad U_4=\gamma A^2a^*
\end{equation}
Se vogliamo quindi ricostruire e osservare l'immagine occorre separare spazialmente queste due componenti del campo trasmesso e per farlo si utilizza uno schema mostrato in Figura \ref{fig:schema} che riprende l'ologramma di Leith-Uptanieks.
 
 \newpage
\section{Apparato sperimentale}
L'apparato a disposizione � riportato in Figura \ref{fig:apparato} ed � composto da:
\begin{enumerate}
\item Laser He-Ne ($\lambda$=633 nm) di potenza $\sim$15 mW
\item Shutter
\item Obbiettivo e filtro spaziale (\emph{pin-hole})
\item Beam splitter
\item Specchi
\item Oggetto
\item Supporto per la lastra fotografica
\item Rivelatore al silicio
\end{enumerate}

\section{Olografia statica}

\subsection{Soldatino con lancia}

\subsection{La caccia}

\section{Olografia dinamica}

\subsection{Cubo con martelletto}

\subsection{Altoparlante}

\section{Lampada spettrale}

\end{multicols}
	
\end{document}