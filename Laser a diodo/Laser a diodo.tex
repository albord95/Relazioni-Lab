\documentclass[a4paper]{article}

\usepackage[T1]{fontenc}
\usepackage[italian]{babel}
\usepackage[latin1]{inputenc}
\usepackage{graphicx}
\usepackage{float}
\usepackage[margin=2 cm]{geometry}
\usepackage{multirow}
\usepackage{multicol}
\author{Alberto Bordin, Giulio Cappelli}
\title{Laser a diodo}
\date{7-8 novembre 2017}

\begin{document}
	\maketitle
	
\begin{abstract}
	Misura della corrente di soglia di un diodo laser per varie temperature di operazione. \\
	Misura della divergenza del fascio. \\
	Misura della dipendenza della lunghezza d'onda dalla temperatura. \\
\end{abstract}

\section{To do}

\begin{multicols}{2}

\section{Teoria}

\section{Apparato sperimentale}
Per poter effettuare le varie misure abbiamo a disposizione:
\begin{itemize}
\item Un laser a diodo della \textit{Hitachi}: modello \textit{HL7812G}
\item Una cella peltier
\item Un \textit{DIGIMASTER DM102}
\item Un supporto per il laser montato su un goniometro con la precisione di 1�
\item
\end{itemize}

\section{Caratteristica P-I}

\subsection{Presa dati}

\subsection{Analisi dati}

\section{Divergenza del fascio}

\subsection{Presa dati}

\subsection{Analisi dati}

\section{Dipendenza $\lambda$ da T}

\subsection{Presa dati}

\subsection{Analisi dati}

\end{multicols}

\end{document}