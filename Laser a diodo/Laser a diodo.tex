\documentclass[a4paper]{article}

\usepackage[T1]{fontenc}
\usepackage[italian]{babel}
\usepackage[latin1]{inputenc}
\usepackage{graphicx}
\usepackage{float}
\usepackage[margin=2 cm]{geometry}
\usepackage{multirow}
\usepackage{multicol}
\author{Alberto Bordin, Giulio Cappelli}
\title{Laser a diodo}
\date{7-8 novembre 2017}

\begin{document}
	\maketitle
	
\begin{abstract}
	Misura della corrente di soglia di un diodo laser per varie temperature di operazione. \\
	Misura della divergenza del fascio. \\
	Misura della dipendenza della lunghezza d'onda dalla temperatura. \\
\end{abstract}

\section{To do}

\begin{multicols}{2}

\section{Teoria}

\section{Apparato sperimentale}
Come si vede in Figura \ref{fig:} per poter effettuare le varie misure abbiamo a disposizione:

\section{Caratteristica P-I}
Analizziamo la dipendenza della corrente di soglia del diodo laser dalla temperatura misurando la caratteristica P-I in tre diverse condizioni: T=12, 25, 45 �C.

\subsection{Presa dati}
Abbiamo misurato la potenza fornita dal diodo laser in funzione della corrente di alimentazione mantenendo la temperatura del diodo costante attraverso l'utilizzo della cella peltier.

Il fascio laser � stato focalizzato sul fotodiodo PD300 e il valore della potenza � stato registrato utilizzando il power meter NOVA RS232.

I valori misurati sono riportati in tabella \ref{tab:} in appendice.

\subsection{Analisi dati}

\section{Divergenza del fascio}

\subsection{Presa dati}

\subsection{Analisi dati}

\section{Dipendenza $\lambda$ da T}

\subsection{Presa dati}

\subsection{Analisi dati}

\end{multicols}

\section{Appendice}
\begin{table}[H]
\centering
\begin{tabular}{cc|cc|cc|cc}
	I [mA] & P [$\mu$W] & I [mA] & P [$\mu$W] & I [mA] & P [$\mu$W] &I [mA] & P [$\mu$W] \\
	\hline
	82.0	&	6230	&	67.3	&	2990	&	50.8	&	64.1	&	43.2	&	34.8	\\
	81.0	&	6030	&	64.8	&	2468	&	51.3	&	69.3	&	41.8	&	32.2	\\
	79.9	&	5800	&	63.0	&	2077	&	52.6	&	89.8	&	39.7	&	28.5	\\
	78.8	&	5480	&	60.5	&	1544	&	49.9	&	57.5	&	38.1	&	26.10 \\
	77.8  &	5290	&	58.5	&	1100	&	49.4	&	54.8	&	36.5	&	23.96 \\
	76.6	&	5050	&	57.3	&	848	&	48.8	&	51.9	&	34.4	&	21.34 \\
	75.3	&	4740	&	56.1	&	578	&	48.01&	48.6	&	31.8	&	18.53  \\
	73.9	&	4430	&	55.1	&	385	&	47.0	&	44.8	&	29.5	&	16.25 \\
	71.0	&	3800	&	54.3	&	218.9&	45.9	&	41.4	&	26.7	&	13.82 \\
	69.1	&	3380	&	53.7	&	142.3&	44.5	&	37.8	&	25.1	&	12.56 \\
	\hline
\end{tabular}
\caption{T = 12�C}
\end{table}


\begin{table}[H]
\centering
\begin{tabular}{cc|cc|cc|cc}
	I [mA] & P [$\mu$W] & I [mA] & P [$\mu$W] & I [mA] & P [$\mu$W] &I [mA] & P [$\mu$W] \\
	\hline
81.9	&	5180	&	72.1	&	3090	&	60.5	&	655	&	47.2	&	38.2	\\
80.8	&	4940	&	71.3	&	2907	&	59.7	&	497	&	45.8	&	35.3	\\
80.3	&	4830	&	69.7	&	2585	&	58.6	&	270.9	&	43.3	&	31.0	\\
79.2	&	4610	&	68.3	&	2267	&	57.6	&	138.5	&	40.5	&	26.80	\\
78.4	&	4420	&	67.2	&	2082	&	56.8	&	101.3	&	38	&	23.61	\\
77.7	&	4280	&	66.3	&	1836	&	55.2	&	72.2	&	35.6	&	20.91	\\
77.2	&	4170	&	65.5	&	1675	&	54.6	&	67.4	&	32.5	&	17.83	\\
76.1	&	3950	&	64.5	&	1471	&	53.6	&	60.3	&	30.6	&	16.14	\\
75.2	&	3750	&	63.1	&	1190	&	52.7	&	55.5	&	28.3	&	14.20	\\
74.7	&	3640	&	62.5	&	1098	&	50.4	&	46.5	&	26.8	&	13.04	\\
73.7	&	3430	&	61.7	&	924	&	48.4	&	40.9	&	25.1	&	11.80	\\
	\hline
\end{tabular}
\caption{T = 25�C}
\end{table}

\begin{table}[H]
\centering
\begin{tabular}{cc|cc|cc|cc}
	I [mA] & P [$\mu$W] & I [mA] & P [$\mu$W] & I [mA] & P [$\mu$W] &I [mA] & P [$\mu$W] \\
	\hline
82.4	&	3760	&	72.8	&	1659	&	64.4	&	128.5	&	52.7	&	40.8	\\
81.1	&	3480	&	71.1	&	1314	&	63.8	&	106.8	&	50.2	&	36.2	\\
80.4	&	3330	&	70.0	&	1077	&	62.9	&	90.0	&	48.8	&	33.9	\\
79.5	&	3130	&	68.8	&	839	&	61.1	&	71.3	&	45.3	&	28.9	\\
77.8	&	2772	&	68.2	&	695	&	60.3	&	66.1	&	43.9	&	27.2	\\
77.4	&	2679	&	67.5	&	558	&	59.5	&	61.4	&	37.8	&	20.63	\\
76.3	&	2431	&	66.8	&	423	&	58.2	&	55.9	&	33.9	&	17.09	\\
75.4	&	2228	&	65.6	&	338	&	57.2	&	52.2	&	29.9	&	13.98	\\
74.5	&	2027	&	65.0	&	168.5	&	55.7	&	48.0	&	25.2	&	10.77	\\
	\hline
\end{tabular}
\caption{T = 43�C}
\end{table}

\begin{table}[H]
\centering
\begin{tabular}{cc|cc|cc|cc}
	$\theta_{//}$& P [$\mu$W] &$\theta_{//}$& P [$\mu$W] & $\theta_{//}$& P [$\mu$W] &$\theta_{//}$& P [$\mu$W] \\
	\hline
0	&	4.30	&	11	&	0.57	&	-3.5	&	2.38	&	-1.5	&	3.56	\\
1	&	4.53	&	12	&	0.33	&	-4	&	2.16	&	-0.5	&	4.10	\\
2	&	4.83	&	14	&	0.16	&	-4.5	&	1.85	&	0.5	&	4.69	\\
3	&	4.59	&	16	&	0.06	&	-5	&	1.54	&	0	&	4.64	\\
4	&	4.26	&	19	&	0.04	&	-5.5	&	1.20	&	1.5	&	4.52	\\
5	&	3.75	&	23	&	0.03	&	-6	&	1.05	&	2.5	&	4.82	\\
5.5	&	3.38	&	30	&	0.02	&	-7	&	0.71	&	2	&	4.95	\\
6	&	2.99	&	45	&	0.02	&	-8	&	0.44	&	2	&	4.96	\\
6.5	&	2.58	&	90	&	0.02	&	-10	&	0.17	&	3.5	&	4.37	\\
7	&	2.30	&	0	&	4.40	&	-12	&	0.07	&	4.5	&	3.89	\\
7.5	&	1.98	&	-1	&	3.73	&	-15	&	0.03	&	-2.5	&	3.19	\\
8	&	1.79	&	-2	&	3.42	&	-30	&	0.03	&	1.5	&	4.44	\\
9	&	1.25	&	-2.5	&	3.05	&	-45	&	0.02	&	2	&	4.61	\\
10	&	0.86	&	-3	&	2.66	&	-85	&	0.02	&	2.5	&	4.27	\\
	\hline
\end{tabular}
\caption{P vs angolo di incidenza parallelo}
\end{table}

\begin{table}[H]
\centering
\begin{tabular}{cc|cc|cc|cc}
	$\theta_{\perp}$& P [$\mu$W] &$\theta_{\perp}$& P [$\mu$W] & $\theta_{\perp}$& P [$\mu$W] &$\theta_{\perp}$& P [$\mu$W] \\
	\hline
-85	&	0.02	&	-17	&	1.73	&	-1	&	4.27	&	14	&	2.51	\\
-70	&	0.02	&	-16	&	1.92	&	0	&	4.71	&	15	&	2.27	\\
-60	&	0.03	&	-15	&	2.08	&	1	&	4.22	&	16	&	2.05	\\
-50	&	0.03	&	-14	&	2.37	&	2	&	4.31	&	17	&	1.82	\\
-45	&	0.10	&	-13	&	2.58	&	3	&	4.38	&	18	&	1.87	\\
-40	&	0.15	&	-12	&	2.73	&	4	&	4.47	&	19	&	1.45	\\
-37	&	0.20	&	-11	&	3.01	&	5	&	4.16	&	21	&	1.21	\\
-34	&	0.28	&	-10	&	2.99	&	6	&	4.15	&	23	&	0.96	\\
-31	&	0.35	&	-11	&	3.04	&	7	&	3.90	&	25	&	0.87	\\
-28	&	0.51	&	-10	&	2.99	&	8	&	3.31	&	27	&	0.62	\\
-25	&	0.70	&	-8	&	3.39	&	9	&	3.72	&	30	&	0.44	\\
-23	&	0.88	&	-6	&	4.12	&	10	&	3.21	&	35	&	0.10	\\
-21	&	1.13	&	-4	&	4.15	&	11	&	3.13	&	45	&	0.11	\\
-19	&	1.37	&	-3	&	3.98	&	12	&	2.88	&	40	&	0.17	\\
-18	&	1.55	&	-2	&	4.23	&	13	&	2.70	&	50	&	0.04	\\
60	&	0.02	&	90	&	0.02	&		&		&		&		\\
	\hline
\end{tabular}
\caption{P vs angolo di incidenza perpendicolare}
\end{table}


\end{document}