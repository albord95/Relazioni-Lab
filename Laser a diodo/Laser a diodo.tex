\documentclass[a4paper]{article}

\usepackage[T1]{fontenc}
\usepackage[italian]{babel}
\usepackage[latin1]{inputenc}
\usepackage{graphicx}
\usepackage{float}
\usepackage[margin=2 cm]{geometry}
\usepackage{multirow}
\usepackage{multicol}
\author{Alberto Bordin, Giulio Cappelli}
\title{Laser a diodo}
\date{7-8 novembre 2017}

\begin{document}
	\maketitle
	
\begin{abstract}
	Misura della corrente di soglia di un diodo laser per varie temperature di operazione. \\
	Misura della divergenza del fascio. \\
	Misura della dipendenza della lunghezza d'onda dalla temperatura. \\
\end{abstract}

\section{To do}

\begin{multicols}{2}

\section{Teoria}

\section{Apparato sperimentale}
Come si vede in Figura per poter effettuare le varie misure abbiamo a disposizione:

\section{Caratteristica P-I}
Analizziamo la dipendenza della corrente di soglia del diodo laser dalla temperatura misurando la caratteristica P-I in tre diverse condizioni: T=12, 25, 45 �C.

\subsection{Presa dati}
Abbiamo misurato la potenza fornita dal diodo laser in funzione della corrente di alimentazione mantenendo la temperatura del diodo costante attraverso l'utilizzo della cella peltier.

Il fascio � stato focalizzato sul fotodiodo attraverso una lente e per ogni valore della corrente abbiamo registrato il corrispondente valore della potenza leggendolo sul power meter NOVA RS232.

Abbiamo effettuato questa misura per tre valori diversi della temperatura del laser a diodo; i valori misurati sono riportati nelle rispettive tabelle in appendice.

\subsection{Analisi dati}

\section{Divergenza del fascio}

\subsection{Presa dati}

\subsection{Analisi dati}

\section{Dipendenza $\lambda$ da T}

\subsection{Presa dati}

\subsection{Analisi dati}

\end{multicols}

\appendix 

\section{Tabelle}

\begin{table}[H]
\centering
\begin{tabular}{|cc|cc|cc|cc|}
	I($\Delta$I) [mA] & P [$\mu$W] & I($\Delta$I) [mA] & P [$\mu$W] & I($\Delta$I) [mA] & P [$\mu$W] &I($\Delta$I) [mA] & P [$\mu$W] \\
	\hline
	82.0(1)	&	6230	&	67.3(1)	&	2990	&	50.8(1)	&	64.1	&	43.2(1)	&	34.8	\\
	81.0(1)	&	6030	&	64.8(1)	&	2468	&	51.3(1)	&	69.3	&	41.8(1)	&	32.2	\\
	79.9(1)	&	5800	&	63.0(1)	&	2077	&	52.6(1)	&	89.8	&	39.7(1)	&	28.5	\\
	78.8(1)	&	5480	&	60.5(1)	&	1544	&	49.9(1)	&	57.5	&	38.1(1)	&	26.10 \\
	77.8(1) 	&	5290	&	58.5(1)	&	1100	&	49.4(1)	&	54.8	&	36.5(1)	&	23.96 \\
	76.6(1)	&	5050	&	57.3(1)	&	848	&	48.8(1)	&	51.9	&	34.4(1)	&	21.34 \\
	75.3(1)	&	4740	&	56.1(1)	&	578	&	48.1(1)	&	48.6	&	31.8(1)	&	18.53  \\
	73.9(1)	&	4430	&	55.1(1)	&	385	&	47.0(1)	&	44.8	&	29.5(1)	&	16.25 \\
	71.0(1)	&	3800	&	54.3(1)	&	218.9&	45.9(1)	&	41.4	&	26.7(1)	&	13.82 \\
	69.1(1)	&	3380	&	53.7(1)	&	142.3&	44.5(1)	&	37.8	&	25.1(1)	&	12.56 \\
	\hline
\end{tabular}
\caption{Valori misurati di corrente e potenza alla temperatura di T = 12�C}
\end{table}


\begin{table}[H]
\centering
\begin{tabular}{|cc|cc|cc|cc|}
	I($\Delta$I) [mA] & P [$\mu$W] & I($\Delta$I) [mA] & P [$\mu$W] & I($\Delta$I) [mA] & P [$\mu$W] &I($\Delta$I) [mA] & P [$\mu$W] \\
	\hline
81.9(1)	&	5180	&	72.1(1)	&	3090	&	60.5(1)	&	655	&	47.2(1)	&	38.2	\\
80.8(1)	&	4940	&	71.3(1)	&	2907	&	59.7(1)	&	497	&	45.8(1)	&	35.3	\\
80.3(1)	&	4830	&	69.7(1)	&	2585	&	58.6(1)	&	270.9&	43.3(1)	&	31.0	\\
79.2(1)	&	4610	&	68.3(1)	&	2267	&	57.6(1)	&	138.5&	40.5(1)	&	26.80\\
78.4(1)	&	4420	&	67.2(1)	&	2082	&	56.8(1)	&	101.3&	38(1)		&	23.61\\
77.7(1)	&	4280	&	66.3(1)	&	1836	&	55.2(1)	&	72.2	&	35.6(1)	&	20.91\\
77.2(1)	&	4170	&	65.5(1)	&	1675	&	54.6(1)	&	67.4	&	32.5(1)	&	17.83\\
76.1(1)	&	3950	&	64.5(1)	&	1471	&	53.6(1)	&	60.3	&	30.6(1)	&	16.14\\
75.2(1)	&	3750	&	63.1(1)	&	1190	&	52.7(1)	&	55.5	&	28.3(1)	&	14.20\\
74.7(1)	&	3640	&	62.5(1)	&	1098	&	50.4(1)	&	46.5	&	26.8(1)	&	13.04\\
73.7(1)	&	3430	&	61.7(1)	&	924	&	48.4(1)	&	40.9	&	25.1(1)	&	11.80\\
	\hline
\end{tabular}
\caption{Valori misurati di corrente e potenza alla temperatura di T = 25�C}
\end{table}

\begin{table}[H]
\centering
\begin{tabular}{|cc|cc|cc|cc|}
	I($\Delta$I) [mA] & P [$\mu$W] & I($\Delta$I) [mA] & P [$\mu$W] & I($\Delta$I) [mA] & P [$\mu$W] &I($\Delta$I) [mA] & P [$\mu$W] \\
	\hline
82.4(1)	&	3760	&	72.8(1)	&	1659	&	64.4(1)	&	128.5&	52.7(1)	&	40.8	\\
81.1(1)	&	3480	&	71.1(1)	&	1314	&	63.8(1)	&	106.8&	50.2(1)	&	36.2	\\
80.4(1)	&	3330	&	70.0(1)	&	1077	&	62.9(1)	&	90.0	&	48.8(1)	&	33.9	\\
79.5(1)	&	3130	&	68.8(1)	&	839	&	61.1(1)	&	71.3	&	45.3(1)	&	28.9	\\
77.8(1)	&	2772	&	68.2(1)	&	695	&	60.3(1)	&	66.1	&	43.9(1)	&	27.2	\\
77.4(1)	&	2679	&	67.5(1)	&	558	&	59.5(1)	&	61.4	&	37.8(1)	&	20.63	\\
76.3(1)	&	2431	&	66.8(1)	&	423	&	58.2(1)	&	55.9	&	33.9(1)	&	17.09	\\
75.4(1)	&	2228	&	65.6(1)	&	338	&	57.2(1)	&	52.2	&	29.9(1)	&	13.98	\\
74.5(1)	&	2027	&	65.0(1)	&	168.5&	55.7(1)	&	48.0	&	25.2(1)	&	10.77	\\
	\hline
\end{tabular}
\caption{Valori misurati di corrente e potenza alla temperatura di T = 43�C}
\end{table}

\begin{table}[H]
\centering
\begin{tabular}{|cc|cc|cc|cc|}
	$\theta_{//}$& P [$\mu$W] &$\theta_{//}$& P [$\mu$W] & $\theta_{//}$& P [$\mu$W] &$\theta_{//}$& P [$\mu$W] \\
	\hline
0.0(5)	&	4.30	&	11.0(5)	&	0.57	&	-3.5(5)	&	2.38	&	-1.5(5)	&	3.56	\\
1.0(5)	&	4.53	&	12.0(5)	&	0.33	&	-4.0(5)	&	2.16	&	-0.5(5)	&	4.10	\\
2.0(5)	&	4.83	&	14.0(5)	&	0.16	&	-4.5(5)	&	1.85	&	0.5(5)	&	4.69	\\
3.0(5)	&	4.59	&	16.0(5)	&	0.06	&	-5.0(5)	&	1.54	&	0.0(5)	&	4.64	\\
4.0(5)	&	4.26	&	19.0(5)	&	0.04	&	-5.5(5)	&	1.20	&	1.5(5)	&	4.52	\\
5.0(5)	&	3.75	&	23.0(5)	&	0.03	&	-6.0(5)	&	1.05	&	2.5(5)	&	4.82	\\
5.5(5)	&	3.38	&	30.0(5)	&	0.02	&	-7.0(5)	&	0.71	&	2.0(5)	&	4.95	\\
6.0(5)	&	2.99	&	45.0(5)	&	0.02	&	-8.0(5)	&	0.44	&	2.0(5)	&	4.96	\\
6.5(5)	&	2.58	&	90.0(5)	&	0.02	&	-10.0(5)	&	0.17	&	3.5(5)	&	4.37	\\
7.0(5)	&	2.30	&	0.0(5)	&	4.40	&	-12.0(5)	&	0.07	&	4.5(5)	&	3.89	\\
7.5(5)	&	1.98	&	-1.0(5)	&	3.73	&	-15.0(5)	&	0.03	&	-2.5(5)	&	3.19	\\
8.0(5)	&	1.79	&	-2.0(5)	&	3.42	&	-30.0(5)	&	0.03	&	1.5(5)	&	4.44	\\
9.0(5)	&	1.25	&	-2.5(5)	&	3.05	&	-45.0(5)	&	0.02	&	2.0(5)	&	4.61	\\
10.0(5)	&	0.86	&	-3.0(5)	&	2.66	&	-85.0(5)	&	0.02	&	2.5(5)	&	4.27	\\	\hline
\end{tabular}
\caption{P vs angolo di incidenza parallelo}
\end{table}

\begin{table}[H]
\centering
\begin{tabular}{|cc|cc|cc|cc|}
	$\theta_{\perp}$& P [$\mu$W] &$\theta_{\perp}$& P [$\mu$W] & $\theta_{\perp}$& P [$\mu$W] &$\theta_{\perp}$& P [$\mu$W] \\
	\hline
-85.0(5)	&	0.02	&	-17.0(5)	&	1.73	&	-1.0(5)	&	4.27	&	14.0(5)	&	2.51	\\
-70.0(5)	&	0.02	&	-16.0(5)	&	1.92	&	0.0(5)	&	4.71	&	15.0(5)	&	2.27	\\
-60.0(5)	&	0.03	&	-15.0(5)	&	2.08	&	1.0(5)	&	4.22	&	16.0(5)	&	2.05	\\
-50.0(5)	&	0.03	&	-14.0(5)	&	2.37	&	2.0(5)	&	4.31	&	17.0(5)	&	1.82	\\
-45.0(5)	&	0.10	&	-13.0(5)	&	2.58	&	3.0(5)	&	4.38	&	18.0(5)	&	1.87	\\
-40.0(5)	&	0.15	&	-12.0(5)	&	2.73	&	4.0(5)	&	4.47	&	19.0(5)	&	1.45	\\
-37.0(5)	&	0.20	&	-11.0(5)	&	3.01	&	5.0(5)	&	4.16	&	21.0(5)	&	1.21	\\
-34.0(5)	&	0.28	&	-10.0(5)	&	2.99	&	6.0(5)	&	4.15	&	23.0(5)	&	0.96	\\
-31.0(5)	&	0.35	&	-11.0(5)	&	3.04	&	7.0(5)	&	3.90	&	25.0(5)	&	0.87	\\
-28.0(5)	&	0.51	&	-10.0(5)	&	2.99	&	8.0(5)	&	3.31	&	27.0(5)	&	0.62	\\
-25.0(5)	&	0.70	&	-8.0(5)	&	3.39	&	9.0(5)	&	3.72	&	30.0(5)	&	0.44	\\
-23.0(5)	&	0.88	&	-6.0(5)	&	4.12	&	10.0(5)	&	3.21	&	35.0(5)	&	0.10	\\
-21.0(5)	&	1.13	&	-4.0(5)	&	4.15	&	11.0(5)	&	3.13	&	45.0(5)	&	0.11	\\
-19.0(5)	&	1.37	&	-3.0(5)	&	3.98	&	12.0(5)	&	2.88	&	40.0(5)	&	0.17	\\
-18.0(5)	&	1.55	&	-2.0(5)	&	4.23	&	13.0(5)	&	2.70	&	50.0(5)	&	0.04	\\
60.0(5)	&	0.02	&	90.0(5)	&	0.02	&		&		&		&		\\	\hline
\end{tabular}
\caption{P vs angolo di incidenza perpendicolare}
\end{table}


\end{document}