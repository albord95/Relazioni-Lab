\documentclass[10pt,a4paper,openbib]{report}
\usepackage[latin1]{inputenc}
\usepackage[italian]{babel}
\usepackage{amsmath}
\usepackage{amsfonts}
\usepackage{amssymb}
\usepackage{subfig}
\usepackage{graphicx}
\usepackage{float}
\usepackage[italian]{varioref}
\usepackage[width=15.00cm, height=24.50cm]{geometry}
\setlength\parindent{0pt}
\setlength{\parskip}{3pt}
\author{A. Bordin, G. Cappelli \\\\ alberto.bordin@sns.it \\ cappelligiulio06@gmail.com}
\title{Laboratorio di Ottica Quantistica}
\date{}
\begin{document}
\maketitle
\tableofcontents

\chapter{Interferometro di Michelson: 06-07 Novembre 2017}
\section{Luned�}

Tolta la lente divergente si allineano i due raggi che non fanno interferenza al finito date le dimensioni dell'apparato.
La vite micrometrica ha un errore del millesimo di millimetro. Gli errori su distanza e numero di picchi sono comparabili (prova a caso 1/35 e 1/55)

\subsection{Cose che dice Tonelli}
\begin{itemize}
	\item camera a vuoto: $5$ cm $\pm 50 \mu m$  
	\item valvole: antiorario per aprire
\end{itemize}


\subsection{Misure HeNe}

picchi 1084
d 0.346 mm

Abbiamo impostato il motion controller a 125 step/sec per un totale di 99999 step per un intervallo di 13:21 min

\section{Marted�}

\subsection{$\lambda$}
Abbiamo impostato il motion controller a 125 step/sec per un totale di 12500 step per un intervallo di 1:40 min
Si notano dei transienti iniziali: 129 picchi e 41 $\mu$m

Su 100 s ne abbiamo perso 1, quindi 1\%

Ora impostiamo 125 step/sec per un totale di 99999 step per un intervallo di 13:33 min

L'errore sulla distanza � pari alla risoluzione della vite, cio� 1 $\mu$m; mentre sul numero di picchi abbiamo considerato $\Delta m$=2 

\begin{table}[H]
\centering
\begin{tabular}{c|ccccccc}
	& m & m* & d  $\mu$m & t mm:ss &$\lambda$ nm & $\Delta\lambda$ & direzione \\
	\hline
	HeNe borcapMM02 &  129 & & 41 & 1:43 & 636 & 18 &  \\ 
	HeNe borcapMM06 &  1088 & 1087.5 & 345 & 13:33 & 634 &2& A\\ 
	HeNe borcapMM07 & 1094 & 1091.5 &345 & 13:43 & 631 & 2 & B\\ 
	HeNe borcapMM08 & 1092 & 1091.5 &345 & 13:36 & 632 & 2 & A\\ 
	650nm borcapMM09 & 1065 & &345 & 13.36 &  648 & 2.2 & A\\ 
	650nm borcapMM10 & 1054 &1052.5 &346 & 13.32 & 657 & 2 & B\\ 
	532nm borcapMM11 & 1306 & 1305.5 &347 & 13.32 & 531 & 2 & A\\
	532nm borcapMM12 & 1293 & 1293.5 &347 & 13.32 & 537 & 2 & B\\
	\hline
	
\end{tabular}
\end{table}

\subsection{Piezo}
Abbiamo fatto una misura della salita da 0 a 100 V e una della discesa da 100 a 0 V contando le frange visivamente e segnando il valore della tensione ogni 10 frange; poi abbiamo ripetuto queste due misure salvando gli andamenti temporali delle frange e della tensione (in ogni acquisizione abbiamo acceso prima il VI Michelson e poi quello del multimetro)

\subsection{Indice di rifrazione}

Per fare il vuoto abbiamo aperto la valvola di collegamento fra camera e tubo e chiuso quella fra tubo ed esterno, quindi abbiamo acceso la pompa per circa 5 min (frange ferme) poi abbiamo azionato il VI e fatto entrare aria prima nel tubo e poi nella camera. A fine acquisizione abbiamo contato i picchi presenti sul grafico.

\begin{itemize}
\item borcapMMn01: 43 picchi (HeNe)
\item borcapMMn02: 43 picchi (HeNe)
\item borcapMMn03: 41 picchi (650 nm)
\item borcapMMn04: 41 picchi (650 nm)
\item borcapMMn05: 51 picchi (532 nm)
\item borcapMMn06: 51 picchi (532 nm)
\end{itemize}

\subsection{Osservazioni}

\begin{itemize}
\item HeNe borcapMM06: si � fermata prima la presa dati rispetto al motorino, probabilmente mancano un paio di picchi quindi aumentiamo la durata dell'acquisizione a 13:43
\item HeNe borcapMM07: si � fermato prima il motorino di circa 10 s, probabilmente sono stati contati dei picchi in pi� quindi diminuiamo la durata dell'acquisizione a 13:36
\end{itemize}

\chapter{Laser a diodo: 09-10 Novembre 2017}
\section{Gioved�}

\subsection{Datasheet}
Scarica e scrivi tutte le cose importanti dei datasheet

\subsection{Cose che dice Tonelli}
10 cm goniometro
non superare 1.5 A peltier



\subsection{Caratteristica P-I al variare di T}

Misuriamo la potenza emessa dal diodo al variare della corrente a tre diverse temperature: 12�C, 25�C, 45�C 

In ognuna delle tre prove scegliamo la lunghezza d'onda letta dal power meter in base a quella riportata sul datasheet del diodo laser: 784 n$\mu$, 786 n$\mu$, 790 n$\mu$

I valori della potenza nei file .txt sono i $\mu$W


\begin{table}[H]
\centering
\begin{tabular}{cc|cc|cc|cc}
	I [mA] & P [mW] & I [mA] & P [] & I [mA] & P [] &I [mA] & P [] \\
	\hline
	82.0 & 6.23 & 71.0 & 3.80  &   &    &   &  \\
	81.0 & 6.03 & 69.1 & 3.38 &   &    &   &   \\
	79.9 & 5.80  & 67.3 & 2.99 &    &   &    &   \\
	78.8 & 5.48 & 64.8 & 2.468 &    &   &    &   \\
	77.8 & 5.29 & 63.0 & 2.077 &    &   &    &   \\
	76.6 & 5.05 & 60.5 & 1.544 &    &   &    &   \\
	75.3 & 4.74  & 58.5 & 1.100 &    &   &    &   \\
	73.9 & 4.43 &  &  &    &   &    &   \\
	
	\hline
	
\end{tabular}
\caption{T = 12 �C}
\end{table}
Abbiamo fatto le altre temperature

\section{Venerd�}
\subsection{Intensit� vs angolo}
Misuriamo l'intensit� relativa ai due piani della giunzione montando il diodo laser su un goniometro 

Leggiamo l'intesit� attraverso un rilevatore collegato ad un multimetro che leggiamo dal pc con DIGITAL MULTIMETER.

Abbiamo allineato il diodo con il polarizzatore e il power meter per� non eravamo soddisfatti quindi lo abbiamo fatto ad occhio usando la cartina e abbiamo controllato col filtro di essere almeno all'interno della sua precisione

Le misure sono fatte a 23�C e con una corrente di controllo di 82.1 $\mu$A

\subsection{$\lambda$ vs T}

Col peltier controlliamo la temperatura del laser. Passiamo lentamente da 12�C a 45�C, intanto controlliamo la lunghezza d'onda del picco letta dallo spettrometro. Abbiamo risoluzione di 1�C e 1nm. Alimentiamo il laser poco sopra soglia $\sim65 mA$

\chapter{Olografia: 13-17 Novembre 2017}

\section{Cose importanti}
\begin{itemize}
\item Cammini ottici uguali
\item Intensit� costante sulla lastra
\item 50\% annerimento lastra per $\lambda$=633 nm quindi 50 $\mu$J/cm$^2$
\item Taratura sensore 
\end{itemize}

\begin{equation}
P[\mu W]=-0.16925+2.6592V[V]
\end{equation}

\begin{equation}
t[s]=\frac{50[\frac{\mu J}{cm^2}]}{P[\mu W]}S[cm^2]
\end{equation}

\section{Olografia statica}
\subsection{soldatino con lancia}

13.12 secondi

\begin{itemize}
\item A = 60 cm
\item B = 92.5 cm
\item C = 78.5 cm
\item D = 38 cm
\item E = 36 cm
\end{itemize}

Abbiamo misurato con il rilevatore al silicio i valori dell'intensit� dei fasci $I_{o}$ e $I_{r}$:

\begin{table}[H]
\centering
\begin{tabular}{|c|c|c|}
	\hline
	  1.14 &  & 1.12  \\
	\hline
	   & 1.48 &   \\
	\hline
	  1.12 &   & 1.13 \\
	\hline
\end{tabular}
\label{tab:tabella1}
\caption{Valori di $I_{r}$ espressi in V con un errore di 0.01 V}
\end{table}

\begin{table}[H]
\centering
\begin{tabular}{|c|c|c|}
	\hline
	  56 &   & 59 \\
	\hline
	   & 59 &   \\
	\hline
	  54 &   & 59 \\
	\hline
\end{tabular}
\label{tab:tabella2}
\caption{Valori di $I_{o}$ espressi in mV con un errore di 1 mV}
\end{table}


\begin{table}[H]
\centering
\begin{tabular}{|c|c|c|}
	\hline
	  1.09 & 1.13  & 1.11  \\
	\hline
	 1.43  & 1.50 &  1.45 \\
	\hline
	  1.21 & 1.26 & 1.24 \\
	\hline
\end{tabular}
\label{tab:tabella3}
\caption{Valori di $I_{tot}$ espressi in V con un errore di 0.01 V}
\end{table}

\begin{table}[H]
\centering
\begin{tabular}{|c|c|c|}
	\hline
	   &   &   \\
	\hline
	   &   &   \\
	\hline
	   &   &   \\
	\hline
\end{tabular}
\label{tab:tabella4}
\caption{Valori di $P_{tot}$ espressi in $\mu$W}
\end{table}

\subsection{caccia}

\begin{itemize}
\item A = 62.5 cm
\item B = 89.5 cm
\item C = 77.5 cm
\item D$_{S}$ = 34.5 cm, D$_{C}$ = 38.5 cm, D$_{P}$ = 50 cm
\item E$_{S}$ = 39 cm, E$_{C}$ = 35.5 cm, E$_{P}$ = 46 cm
\end{itemize}

\begin{table}[H]
\centering
\begin{tabular}{|c|c|c|}
	\hline
	  1.19 & 1.24  &  1.19 \\
	\hline
	  1.50 & 1.54  & 1.49  \\
	\hline
	  1.20 & 1.22 & 1.19  \\
	\hline
\end{tabular}
\caption{Valori di $I_{r}$ espressi in V con un errore di 0.01 V}
\end{table}

\begin{table}[H]
\centering
\begin{tabular}{|c|c|c|}
	\hline
	  67 & 69  & 70  \\
	\hline
	  68 & 70 &  71 \\
	\hline
	  69 & 68 & 71 \\
	\hline
\end{tabular}
\caption{Valori di $I_{o}$ espressi in mV con un errore di 1 mV}
\end{table}

\begin{table}[H]
\centering
\begin{tabular}{|c|c|c|}
	\hline
	  1.25 & 1.30  & 1.25 \\
	\hline
	  1.56 & 1.59 & 1.52 \\
	\hline
	 1.25  & 1.27 & 1.24 \\
	\hline
\end{tabular}
\label{tab:tabella4}
\caption{Valori di $I_{tot}$ espressi in V con un errore di 0.01 V}
\end{table}

\begin{table}[H]
\centering
\begin{tabular}{|c|c|c|}
	\hline
	   &   &   \\
	\hline
	   &   &   \\
	\hline
	   &   &   \\
	\hline
\end{tabular}
\label{tab:tabella4}
\caption{Valori di $P_{tot}$ espressi in $\mu$W}
\end{table}

\section{Olografia dinamica}

\subsection{Cubo con martelletto}
Tempo stimato di 12.31 s di cui 6 con martelletto premuto e 6 senza. Abbiamo preparato la lastra, premuto il martelletto, acceso il laser, lasciato il martelletto, spento il laser.

\begin{itemize}
\item A = 62.5 cm
\item B = 89.5 cm
\item C = 78 cm
\item D = 39 cm
\item E = 35 cm
\end{itemize}

\begin{table}[H]
\centering
\begin{tabular}{|c|c|c|}
	\hline
	 1.24  & 1.29 & 1.25 \\
	\hline
	  1.53 & 1.55 & 1.51 \\
	\hline
	  1.19 & 1.24 & 1.15 \\
	\hline
\end{tabular}
\caption{Valori di $I_{r}$ espressi in V con un errore di 0.01 V}
\end{table}

\begin{table}[H]
\centering
\begin{tabular}{|c|c|c|}
	\hline
	 51  & 51 & 51 \\
	\hline
	  53 & 52 & 53 \\
	\hline
	  50 & 50 & 50 \\
	\hline
\end{tabular}
\caption{Valori di $I_{o}$ espressi in mV con un errore di 1 mV}
\end{table}

\begin{table}[H]
\centering
\begin{tabular}{|c|c|c|}
	\hline
	  1.23 & 1.30  & 1.25 \\
	\hline
	  1.55 & 1.58 & 1.54 \\
	\hline
	  1.25 & 1.30 & 1.22 \\
	\hline
\end{tabular}
\label{tab:tabella4}
\caption{Valori di $I_{tot}$ espressi in V con un errore di 0.01 V}
\end{table}

\begin{table}[H]
\centering
\begin{tabular}{|c|c|c|}
	\hline
	   &   &   \\
	\hline
	   &   &   \\
	\hline
	   &   &   \\
	\hline
\end{tabular}
\label{tab:tabella4}
\caption{Valori di $P_{tot}$ espressi in $\mu$W}
\end{table}

\subsection{Altoparlante}

Abbiamo tenuto acceso l'altoparlante alla frequenza di 783 Hz (un Sol5) segnata 1.03 sul generatore di funzioni.

Analogamente a quanto fatto per il martelletto abbiamo tenuto acceso l'altoparlante per le met� del tempo totale di 11.7 s stimato e 11.43 s effettivo.

\begin{itemize}
\item A = 63 cm
\item B = 87.5 cm
\item C = 69 cm
\item D = 36.5 cm
\item E = 44 cm
\end{itemize}

\begin{table}[H]
\centering
\begin{tabular}{|c|c|c|}
	\hline
	  1.24 & 1.31 & 1.28 \\
	\hline
	 1.58 & 1.61 & 1.57 \\
	\hline
	  1.26 & 1.33 & 1.25 \\
	\hline
\end{tabular}
\caption{Valori di $I_{r}$ espressi in V con un errore di 0.01 V}
\end{table}

\begin{table}[H]
\centering
\begin{tabular}{|c|c|c|}
	\hline
	  98 & 95 & 96 \\
	\hline
	  94 & 97 & 100 \\
	\hline
	  93 & 97 & 99 \\
	\hline
\end{tabular}
\caption{Valori di $I_{o}$ espressi in mV con un errore di 1 mV}
\end{table}

\begin{table}[H]
\centering
\begin{tabular}{|c|c|c|}
	\hline
	  1.34 & 1.38 & 1.36 \\
	\hline
	  1.62 & 1.67 & 1.65 \\
	\hline
	  1.29 & 1.38 & 1.30 \\
	\hline
\end{tabular}
\label{tab:tabella4}
\caption{Valori di $I_{tot}$ espressi in V con un errore di 0.01 V}
\end{table}

\begin{table}[H]
\centering
\begin{tabular}{|c|c|c|}
	\hline
	   &   &   \\
	\hline
	   &   &   \\
	\hline
	   &   &   \\
	\hline
\end{tabular}
\label{tab:tabella4}
\caption{Valori di $P_{tot}$ espressi in $\mu$W}
\end{table}

\chapter{Fibre Ottiche: 20-24 Novembre 2017}

\section{Procedimento da seguire}
Abbiamo allineato il laser all'interno di un foro cilindrico in un cubo metallico.

\begin{enumerate}
\item Inserire coso nero
\item Spellare la fibra meccanicamente o chimicamente
\item Pulire la fibra con l'alcool
\item Tagliare la fibra col taglierino
\item Inserire il portafibra 
\item Controllare il taglio
\item Mettere la fibra al sicuro
\end{enumerate}

Le dimensioni del sensore del power meter sono 7 mm

Abbiamo un fondo del sensore di 2 nW con un errore del 5\% confrontabile con quello dovuto alla distanza fra il sensore e l'apertura 
che � 23.8 cm

-6 = 40 nW

Lo zero del goniometro � a 13�, luce ambientale 2 nW

Massimo fra 13 e 14� a 2.65 muW

 d$_{2}$ = 11.1 cm
 
P$_{out}$(max) = 9.81 a 13�

Fondo scala 4 nW a -16�

P$_{out}$(max) = 12 muW fra 8 e 9 ad una distanza di 8.9 cm

Abbiamo degli spot scuri e chiari sullo schermo di dimensioni tipiche di 1 cm$^2$ ad una distanza di 105 cm dalla fibra
\subsection{Attenuazione}
\subsubsection{HeNe}
Abbiamo srotolato dalla bobina di fibra multimodo F-MLD 2 m di fibra ed utilizzando un lanciatore con una lente col fuoco a 2 mm abbiamo allineato il laser ad HeNe nella fibra e abbiamo letto col power meter la potenza uscente:

P$_{out}$ = 598 $\mu$W

Adesso tagliamo i 2 m iniziali da cui (lasciando tutto invariato) leggiamo la potenza entrante:

P$_{in}$ = 1.63 mW

Abbiamo utilizzato uno scramble per eliminare i modi spuri dalla fibra ad una distanza di 41 cm. 

Lunghezza finale fibra = 205.3 cm

Abbiamo tolto circa 210 cm

Ci dimentichiamo sempre di infilare il cosino nero!

\subsubsection{verde}

P$_{out}$ = 156 $\mu$W

Adesso tagliamo i 2 m iniziali da cui (lasciando tutto invariato) leggiamo la potenza entrante:

P$_{in}$ = 1.47 mW

\subsection{Attenuazione 2}
Misuriamo la potenza uscente dai 2 m di fibra tagliati la volta scorsa P = 3.41 mW e poi stringiamo lo scramble per vedere quando si rompe la fibra. Si � rotta intorno a qualche centinaio di $\mu$W e la potenza � scesa a qualche nW (si � sentito il rumore di quando si rompe).

Abbiamo inserito la fibra nel lanciatore e ottenuto una potenza uscente dai $\sim$300 m di fibra di 1.66 mW per l'He-Ne e di 830 $\mu$W per il laser verde prima dello scramble (posto a circa 30 cm dal lanciatore). Stringiamo adesso le scramble per eliminare i modi spuri e misuriamo la potenza uscente, poi tagliamo la fibra e misuriamo la potenza entrante e da questi valori otteniamo l'attenuazione della fibra definita da:
\begin{equation}
\Gamma = \frac{10}{L} \text{log}\left |\frac{P_{in}}{P_{out}} \right |
\end{equation}

\begin{table}[H]
\centering
\begin{tabular}{cccc}
	\hline
	   & $P_{out}$ [$\mu$W]& $P_{in}$ [mW]& $\Gamma [dB]$ \\
	\hline
	He-Ne  & 832 & 2.44 &  \\
	 532 nm & 320(2) & 2.80(3) &  \\
	\hline
\end{tabular}
\end{table}

La cosa importante � che fra le due misure non varino le condizioni sperimentali!!!

\subsection{Propagazione di un modo LP$_{01}$}
Abbiamo spellato chimicamente due metri di1 fibra a singolo modo F-SV (4,3/125/245) e le iniettiamo il laser HeNe.
Poi attraverso il power meter con attaccato uno schermo con una fessura prendiamo le misure della potenza in funzione dell'angolo per verificare che si sia propagato il modo TEM$_{00}$ (fit ad una gaussiana).

Affinch� si propaghi solo il modo LP$_{01}$ si deve avere 
\begin{equation}
V = \frac{2\pi}{\lambda_{0}}\cdot N.A.\cdot a \le 2.405 
\end{equation}

Abbiamo una potenza massima di 550 $\mu$W senza fenditura.

Luce ambientale 2 nW

I dati sono salvati in $\mu$W.

errori per la gaussiana: ultimo digit di errore di digitalizzazione e almeno 50 nW di sballonzoli

\subsection{Propagazione modi superiori}
Adesso vogliamo osservare la propagazione dei modi superiori e per farlo utilizziamo una fibra a singolo modo con $\lambda_{0}$=1300 nm cos� che si abbia 
x
\begin{equation}
V = \frac{2\pi}{\lambda_{0}}\cdot N.A.\cdot a \ge 2.405 
\end{equation}

e quindi variando l'allineamento si propagano i diversi modi a seconda dell'angolo di ingresso nella fibra.

Abbiamo spellato la fibra F-SMF-28 (9,3/125/245) chimicamente e una volta tagliata abbiamo allineato un capo al laser e l'altro ad un obiettivo x20 in modo da poter proiettare l'immagine dei modi su uno schermo e fotografarli.

Dato che la frequenza normalizzata per il laser verde a 532 nm ha un valore molto alto e quindi un elevato numero di modi indipendenti abbiamo provato ad usare anche l'HeNe.

\subsection{Fibra a conservazione di polarizzazione}
Abbiamo spellato chimicamente alle estremit� e al centro una fibra F-SPV SM (3,2/125/245) al fine di misurarne in battimenti.
Questo tipo di fibra ha un core stressato meccanicamente che presenta una forma ellittica e a ciascun asse � associato un diverso indice di rifrazione. 

Inseriamo la fibra nel lanciatore e osserviamo l'ellisse sullo schermo bianco all'uscita, in questo modo, attraverso un goniometro fissato sul portafibra fissiamo le posizione degli assi (90�-30�) e quindi scambiamo le estremit� e facciamo in modo che la fibra sia a 45� cos� da vedere zone scure e chiare al centro della fibra.

\subsection{Lente di GRIN}
Con una lente di GRIN misuriamo l'accoppiamento  della fibra multimodo con un laser a diodo e un LED entrambi a 830 nm. Per farlo misuriamo la potenza in uscita dal diodo laser e dal LED $P_{in}$ e quella in uscita dalla fibra $P_{out}$, una volta allineato il fascio al suo interno con l'utilizzo della lente di GRIN.

Il coefficiente di accoppiamento � quindi definito da:

\begin{equation}
\Gamma = 10 \text{log}\left |\frac{P_{in}}{P_{out}} \right |
\end{equation}

\begin{table}[H]
\centering
\begin{tabular}{ccccc}
	\hline
	   & $I_{in}$ [mA]  & $P_{in}$ [mW]& $P_{out}$ [mW]& $\Gamma [dB]$ \\
	\hline
	 Laser  &  78.0(1) & 2.10(1) &  6.12(10) & 4.65\\
	 Laser  & 78.3(2) & 2.67(1) &   &\\
	 Laser  & 78.4(2) & 3.41(1) &   &\\
	 Laser  & 78.3(2) & 3.52(1) &   &\\
	 Laser  & 78.0(2) & 3.48(1) &   &\\
	 Laser  & 78.0(1) & 3.50(1) &   6.19(6) &\\
	\hline
	  LED & 81.0(2)  & 3.90(1) [$\mu$W] & 8.21(10) & \\
	  LED & 81.1(2)  & 3.96(1) [$\mu$W] & 8.21(10) & \\
	  LED & 81.1(2)  & 4.45(1) [$\mu$W] & 8.21(10) & \\
	  LED & 81.1(2)  & 4.78(1) [$\mu$W] & 8.21(10) & \\
	  LED & 81.1(2)  & 4.91(2) [$\mu$W] & 8.21(10) & \\
	\hline
\end{tabular}
\end{table}

accorgimento: notiamo che l'alimentatore di laser e led ha delle fluttuazioni 1/f non trascurabili. Quindi prestiamo attenzione, regolando la manopola, ad alimentare sempre con la stessa corrente.

\subsection{suono}

sentiamo i 98 Hz della luce ambientale (2*50 Hz perch� abbiamo lampade al neon) misurati con l'accordatore del telefono

\chapter{Analizzatore di spettro: 27-28 Novembre2017}
\section{Venerd�}
generatore di funzioni:\\
frequenza = 4.50 Hz\\
ampiezza 4 V picco picco

impostiamo ampiezza e offset del piezoelettrico circa a met� del fondoscala nella speranza di essere nella zona lineare del piezoelettrico del Fabri-Perot

\subsection{presa dati}
impostiamo l'ampiezza del piezoelettrico in modo da avere 3 ordini di interferenza del Fabri-Perot.
In principio stiamo nel fronte di salita dell'onda triangolare del generatore di funzioni

le misure seguono il seguente algoritmo:\\
stepA) prendiamo la misura dei tre ordini per misurare la FSM\\
stepB) zoommiamo il fondoscala dell'oscilloscopio (senza modificare nient'altro) per ingrandire il picco centrale dell'ordine centrale e misurare la finezza

iteriamo gli step A e B 5 volte
impostiamo il livello del trigger sulla rampa in modo da non dover spostare l'asse del tempo passando da step A a step B

poi ci spostiamo sulla discesa della rampa e riapplichiamo l'algoritmo

in caso di necessit� prima dello step A aggiustiamo il setting toccando SOLO la manopola dell'offset del piezoelettrico e in caso l'allineamento degli specchi

Ho cambiato l'allineamento del laser dopo ogni misura per cercare di rendere pi� simmetrica la curva


\end{document}