\documentclass[a4paper]{article}

\usepackage[T1]{fontenc}
\usepackage[italian]{babel}
\usepackage[latin1]{inputenc}
\usepackage{graphicx}
\usepackage{float}
\usepackage[margin=2 cm]{geometry}
\usepackage{multirow}
\usepackage{multicol}
\usepackage{textcomp}
\usepackage{caption}
\author{A. Bordin, G. Cappelli}
\title{Fibre ottiche}
\date{20-24 Novembre 2017}
\newcommand{\minitab}[2][l]{\begin{tabular}#1 #2\end{tabular}}


\begin{document}
	\maketitle
	
	\begin{abstract}
		 
	\end{abstract}
	
\section{Teoria}

\section{Apparato sperimentale}

\section{Apertura numerica}

\subsection{Teoria}

\subsection{Presa dati}

2 tabelle

\subsection{Analisi dati}

2 plot con interpolazione quadratica o al massimo cubica

\section{Attenuazione - albe}

\subsection{Teoria}

\subsection{Presa dati}

\subsection{Analisi dati}

4 plot: 1 normale e 1 loglog per 2 volte

\section{Propagazione modo LP$_{01}$ in una fibra SM}

\subsection{Teoria}

\subsection{Presa dati}

1 tabella

\subsection{Analisi dati}

1 plot

\section{Propagazione modi superiori}

\subsection{Teoria}

\subsection{Presa dati}

\subsection{Analisi dati}

\section{Fibra a conservazione di polarizzazione}

\subsection{Teoria}

\subsection{Presa dati}

\subsection{Analisi dati}

1 figura

\section{Lente di GRIN - albe}

\subsection{Teoria}

\subsection{Coefficiente di accoppiamento}

\subsubsection{Presa dati}

1 tabella

\subsubsection{Analisi dati}

\subsection{Trasmissione di un segnale}
	
\end{document}